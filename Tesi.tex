%!TEX TS-program = pdflatex

\documentclass[10pt,a4paper,italian]{article} %classe book, A4 con pt 10

\usepackage[utf8]{inputenc} %codifica il documento con UTF8
\usepackage[italian]{babel}
\usepackage{amsmath} %Arricchisce la scelta nel comporre le formule.
\usepackage{amssymb} %idem
\usepackage{graphicx} %facilita la gestione delle figure
\usepackage{amsfonts}
\usepackage{float} %metto le immagini dove voglio
\usepackage{pdfpages} %per il frontespizio
\usepackage[bottom]{footmisc} %note correttamente a pie di pagina
\usepackage[binary-units=true]{siunitx} %unità di misura
\usepackage{textcomp} %marchio registrato, riservato
\usepackage{minted} %comandi bash
\usepackage{eurosym} %simbolo dell'euro

\graphicspath{ {images/} } %le immagini sono nella cartella images
\pagestyle{plain} %lascia vuota la testata e mette il numero di pagina centrato nel piede

\author{Stefano Orioli}
\title{Sistemi flessibili per la cattura di traffico bluetooth}
\date{14/02/2018}




\begin{document}

%\maketitle %stampo il titolo

\includepdf{frontespizio}

\newpage 

\tableofcontents %stampo l'indice

%!TEX root = Tesi.tex

\section{Introduzione}
Il Bluetooth è una tecnologia di comunicazione a corto raggio che esiste ormai da parecchi anni sul mercato, ma con l'avvento della specifica Low Energy, introdotta nel 2010 nella versione 4.0 e caratterizzata da un notevole risparmio in termini energetici, è stata adottata da una gran quantità di dispositivi, anche destinati a usi differenti. Basti pensare a tutti quei dispositivi che si definiscono \lq Smart\rq, dalle Televisioni agli Smartphone, dai PC agli orologi, la tecnologia ha un grande impatto sulla nostra vita di tutti i giorni.
Questa rivoluzione comunicativa prende il nome di IoT\footnote{Internet of Things, ovvero l'internet delle cose.}, il mondo dei dispositivi fisici, dove anche il più piccolo di essi, con capacità di calcolo limitate, è in grado di comunicare la propria presenza ed ottenere informazioni sulla rete proprio grazie a queste nuove tecnologie.

\'E quindi corretto chiedersi: queste tecnologie sono sicure? I danni che un malintenzionato sarebbe capace di provocare se riuscisse a manipolare a piacimento il comportamento di questi dispositivi, sono innumerevoli;
Basti pensare ad una serratura intelligente che sblocca la porta di casa quando il proprietario si trova di fronte ad essa, e se non fosse il proprietario quello davanti alla porta, ma tramite un dispositivo di ritrasmissione si spacciasse per esso? Lo stesso discorso si può applicare alle serrature ed al sistema di avvio delle recenti automobili dotate di sistema Keyless\footnote{Senza chiavi, permette l'apertura e l'avvio dell'automobile semplicemente tenendo in tasca le chiavi.}.
Restando in campo informatico, sempre più aziende stanno adottando soluzioni Smart per accedere ai PC, senza dover inserire manualmente la classica password; Apple da la possibilità ai propri utenti di sbloccare il  MacBook tramite il loro Apple Watch, semplicemente avvicinandolo allo schermo; un malintenzionato riuscirebbe con facilità a rubare i dati personali di un utente se riuscisse a sfruttare una falla di questo meccanismo di sblocco.

\'E proprio su questi presupposti che si basa questo progetto di tesi, testare la sicurezza del protocollo Bluetooth tramite la creazione di uno Sniffer a basso costo che permetta di visualizzare ed analizzare i pacchetti scambiati da due dispositivi durante tutta la durata di una connessione. L'implementazione di uno Sniffer è utile anche per un'attività di Debug nello sviluppo delle applicazioni; permette di vedere realmente la composizione del pacchetto inviato e quindi di trovare facilmente errori nell'applicativo.
Uno degli sviluppi futuri è quello riuscire a creare un ripetitore di segnale tra due punti, basandosi sul lavoro svolto in questa tesi, che faccia credere ai dispositivi di essere a stretto contatto quando in realtà li separa una distanza maggiore; questo dimostrerebbe che un dispositivo Bluetooth Low Energy è vulnerabile ad attacchi di tipo Relay, e forzerebbe i costruttori a risolvere queste vulnerabilità e quindi a migliorare la sicurezza del protocollo.

\newpage 

\subsection{Alternative}
Sul mercato esistono già prodotti di sniffing completi, funzionanti e pronti per essere usati; il problema di questi dispositivi è che sono costosi e per la maggior parte non Open Source. La principale limitazione di ciò è che il codice su cui si basano non può essere manipolato a piacimento secondo le proprie esigenze e quindi questi programmi possono essere unicamente usati per lo scopo per cui sono stati creati. Se uno sviluppatore in possesso di uno di questi dispositivi a Closed Source volesse cambiare un leggero aspetto di un programma di Sniffing preconfezionato dovrebbe riscriversi completamente il codice da zero, con un enorme dispendio di tempo e risorse per simulare il comportamento di un applicativo già esistente.

\subsection*{Ubertooth One}
Ubertooth One è un progetto Open Source di sviluppo su reti Wireless, capace di comunicare utilizzando il protocollo trasmissivo BLE. Sulla loro pagina GitHub \cite{ubertooth_github_web} è possibile reperire, oltre ai vari progetti sviluppati appositamente, anche il disegno delle componentistiche Hardware, per poter assemblarlo e costruirlo privatamente o addirittura modificarlo secondo le proprie necessità.

\begin{figure}[H]
\includegraphics[width=340pt]{ubertooth_one}
\centering
\caption{UberTooth One}
\end{figure}

Si hanno ha disposizione vari progetti creati per questo dispositivo, tra i tanti va citato \lq Bluetooth Captures in PCAP\rq  \cite{ubertooth_pcap_web} che permette di ascoltate, in una sorta di modalità promiscua, tutte le comunicazioni bluetooth su un certo canale, oppure tramite l'ascolto di una CONNECT\_ REQ. seguire una connessione. Permette ulteriormente di interfacciarsi con l'applicazione \emph{WIRESHARK}\footnote{\'E un software per analisi di protocolli o pacchetti utilizzato per la soluzione di problemi di rete, per l'analisi e lo sviluppo di protocolli o di software di comunicazione.} per visualizzare in tempo reale il traffico BLE, applicare filtri su indirizzi o potenza del segnale ed inviare pacchetti personalizzati.

L'aspetto negativo di questa soluzione sta nel costo, difatti ad oggi si trova in commercio ad un prezzo che si aggira attorno ai 150\euro  + spese di consegna \cite{ubertooth_reseller_web}.

\subsection*{Bluefruit LE sniffer}
Il Bluefruit LE monta al suo interno il chip della Nordic nRF51822 che viene venduto già programmato con un software che trasforma il dispositivo in uno sniffer di traffico Bluetooth Low Energy. I pacchetti raccolti vengono mandati in automatico a Wireshark  dove possono essere visualizzati con un'opportuna struttura che facilita la comprensione delle varie parti del pacchetto, senza far riferimento al manuale Bluetooth Core.

\begin{figure}[H]
\includegraphics[width=320pt]{bluefruit_sniffer}
\centering
\caption{Bluefruit LE Sniffer - Bluetooth Low Energy (BLE 4.0) - nRF51822}
\end{figure}

Le limitazioni di questa soluzione sono molteplici; il software che viene installato per gestire il dispositivo come sniffer non è open source; il tool che permette l'interfacciamento tra il dispositivo e wireshark esiste solo per l'ambiente windows. Il dispositivo così come è venduto non può essere programmato, necessita infatti di un programmatore esterno, come il J-Link venduto ad un costo che si aggira sui 15\euro .
Il costo del dispositivo è di circa 25\euro\ paragonabile al prezzo di acquisto di un RedBear Nano2.

\newpage 

\subsection*{nRF52 DK}
La board di sviluppo nRF52 DK della Nordic Semiconductor è una periferica di sviluppo versatile e completa per testare applicazioni BLE, ANT e protocolli proprietari che usano le frequenze 2.4GHz. Integra 4 led e 4 bottoni che possono essere usati rispettivamente per ricevere informazioni sullo stato di funzionamento del dispositivo e per impartire comandi a quest'ultimo.

\begin{figure}[H]
\includegraphics[width=320pt]{nrf52_dk}
\centering
\caption{Board di sviluppo NRF52\_ DK creata dalla società Nordic Semiconductor.}
\end{figure}

Essendo ufficialmente supportata dalla stessa Nordic, lo sviluppatore ha a disposizione molti tool per lo sviluppo e il testing di applicazioni. Esiste anche un software di sniffing preconfezionato che permette di svolgere tutte le operazioni di sniffing direttamente da wireshark, tra cui la possibilità di seguire in maniera del tutto automatizzata una connessione e vedere in tempo reale la composizione e il significato di tutti i pacchetti scambiati. 
Sfortunatamente questo software esiste solo per windows e non è open source, quindi non è utilizzabile per lo scopo di questa tesi.
Oltretutto il costo di questa board di sviluppo è di circa 40\euro\ più spese di consegna, più alto di quello di un Nano2.

\subsection*{USRP EttusB210}
Il B210 fornisce una singola scheda integrata su una piattaforma USRP\footnote{Universal Software Radio Peripheral, componente della Ettus Research di tipo Software-defined radio (SDR)} con una copertura continua di frequenza da 70 Mhz a 6 GHz. Progettato per la sperimentazione a basso costo, combina il ricetrasmettitore di conversione diretta RFC AD9361 che fornisce fino a 56 MHz di larghezza di banda in tempo reale, un FPGA\footnote{Field Programmable Gate Array, circuito integrato le cui funzionalità sono programmabili via software}, Spartan6 XC6SLX150 open-source e riprogrammabile, ed una connettività ad alta velocità utilizzando USB 3.0 con canale di alimentazione dedicato. Il B210 consente un facile e veloce sviluppo con il toolkit di sviluppo software GNURadio, consentendo così di sperimentare con un'ampia gamma di segnali e bande di frequenza, inclusi trasmissioni FM, TV, cellulari, Wi-fi e Bluetooth.

\begin{figure}[H]
\includegraphics[width=340pt]{usrp_b210}
\centering
\caption{USRP Ettus B210, Ettus Research una società della compagnia National Instrument. }
\end{figure}

La massima banda trasmissiva campionabile in tempo reale con questo dispositivo è di 56 MHz a 61,44 MS/s\footnote{Mega Sample al secondo, unità di misura della frequenza di campionamento.} inferiore alla banda trasmissiva del Bluetooth Low Energy. Il costo di questa board è molto elevato rispetto alle precedenti soluzioni, essendo di circa 1.250\euro\ .
Per questo progetto disponiamo comunque di due di questi dispositivi, essendo stati acquistati dall'università in precedenza, che riescono a coprire gli 80 MHz di banda trasmissiva BLE. Si sono rilevati molto utili per capire ciò che realmente passa nell'etere a quelle frequenze. Il dispositivo campiona qualsiasi segnale nella banda trasmissiva, creando una gran quantità di dati che devono essere poi analizzati e filtrati con un software creato appositamente per estrarne i pacchetti BLE di nostro interesse.








%!TEX root = Tesi.tex

\section{Bluetooth Low Energy}
\subsection{Introduzione}
Il Bluetooth Low Energy (BLE) è una tecnologia wireless introdotta nello standard 4.0, disegnata e commercializzata dal SIG\footnote{Bluetooth Special Interest Group: organizzazione che regolamenta e definisce gli standard  per la trasmissione dati tramite la tecnologia Bluetooth.}; ideata per trasmissioni di dati di breve dimensione, essa si differenzia dal Bluetooth classico per un inferiore consumo di energia, mantenendo la stessa portata trasmissiva.

\subsection{Banda Trasmissiva}
BLE opera alla stessa banda di frequenza del bluetooth classico, 2,400 - 2,4835 GHz, ma utilizza un numero inferiore di canali, 40 canali da 2 MHz l'uno.
Nel canale trasmissivo i dati sono trasmessi con una modulazione GFSK.\footnote{Gaussin Frequency Shift Keying, limita la larghezza dello spettro trasmissivo tramite un filtro di tipo Gaussiano.}
I bitrate trasmissivi sono limitati ai valori 125 kbit/s, 1 Mbit/s o 2 Mbit/s.

\subsubsection{Mapping tra canali e frequenze}
Il protocollo BLE utilizza 40 canali fisici differenti per la trasmissione, con una numerazione che va da 0 a 39. Ogni canale ha una frequenza centrata in $2402 + (k * 2)$ MHz, dove k assume il valore del canale considerato.
Questi 40 canali si dividono in:
\begin{itemize}
\item 3 canali 0, 12 e 39 centrati nelle frequenze 2402 MHz, 2426 MHz e 2480 MHz ed utilizzati per l'Advertise. La distribuzione di questi canali non è casuale: essi difatti sono i canali utilizzati da un dispositivo per far conoscere la propria presenza ai vicini, è quindi fondamentale che almeno uno di essi sia disponibile per la trasmissione, quindi privo di disturbi generati da altri sistemi che operano nella stessa banda trasmissiva. Sono stati distribuiti lontani tra di loro cercando di utilizzare tutta la banda disponibile per massimizzare la possibilità di avere un canale libero per la trasmissione. Un altro motivo della scelta è stato quello di poter facilitare la co-esistenza con la tecnologia Wifi, che utilizza la stessa banda per trasmettere, centrata principalmente nei canali Wifi 1, 6 e 11.
\item i 37 canali restanti sono utilizzati per la trasmissione di dati che vengono utilizzati solo a connessione avvenuta per lo scambio di informazioni.
\end{itemize} 


Per una miglior leggibilità ed un più facile utilizzo, si è scelto di mappare i canali reali con dei corrispondenti indici fittizi che vedono nelle numerazioni da 0 a 36 i canali data e 37, 38, 39 per i canali di Advertise;
\begin{itemize}
\item[-] canale fisico 0, di Advertise a frequenza 2402 MHz, mappato come canale 37.
\item[-] canale fisico 12, di Advertise a frequenza 2426 MHz, mappato come canale 38.
\item[-] canale fisico 39, di Advertise a frequenza 2480 MHz, mappato come canale 39. 
\item[-] canale fisico 1, di scambio dati a frequenza 2404 MHz, mappato come canale 0.
\item[-] \dots
\item[-] canale fisico 38, di scambio dati a frequenza 2478 MHz, mappato come canale 36.
\end{itemize}

% figura canali

% figura canali con i 3 wifi



Bluetooth Low Energy utilizza il \emph{Frequency Hopping}, una tecnica trasmissiva che consiste nel cambiare canale trasmissivo per ogni pacchetto inviato durante una trasmissione; l'utilizzo di questa tecnica permette di ridurre i problemi di interferenza con gli altri canali ed allo stesso tempo di aumentare la sicurezza delle trasmissioni.

\subsection{Stack protocollare}
Storicamente lo stack Bluetooth è diviso in 2 componenti fondamentali: il \emph{Controller} e l'\emph{Host}.
La parte di Controller è quella che si occupa di eseguire le componenti di basso livello dello stack necessarie per gestire i pacchetti scambiati a livello fisico e le loro temporizzazioni; la parte di Host comprende le componenti di alto livello tra cui profili e API\footnote{Application Program Interface: insieme di procedure utili a svolgere uno specifico compito}. La parte di Host, diversamente da quella di controllo, astrae dall'hardware ed è caratterizzata da una gestione meno rigida delle temporizzazioni.

L'\emph{HCI}\footnote{Host Controller Interface}, l'interfaccia tra Host e Controller si occupa di mettere in comunicazione queste due componenti fondamentali, utilizzando canali comunicativi quali UART o USB.
Se i due componenti sono montati su uno stesso chip, come nel caso dei SoC\footnote{System on a Chip: sistema completamente contenuto su un solo Chip}, allora l'interfaccia HCI è opzionale e può essere omessa.

\includegraphics[scale=0.5]{stack_ble.png}

\begin{itemize}
\item Physical Layer (PHY): è il livello fisico, ovvero l'etere; lavora nella banda di frequenza 2,4 GHz ISM\footnote{Industrial, Scientific and Medical: frequenze riservate alle applicazioni di radiocomunicazioni non commerciali, ovvero per uso industriale, scientifico e medico.} Si compone di 40 canali da 2 MHz ognuno suddivisi in 3 per advertising e 37 per lo scambio dati.

\item Link Layer (LL): si occupa della gestione della sequenza e della temporizzazione dei pacchetti scambiati. Dialoga con i nodi vicini e scambia informazioni quali i parametri di connessione e controllo di flusso.\linebreak 
\'E una macchina a stati che si compone di 5 stati: Standby, Advertising, Scanning, Initiating, Connection. 

\includegraphics[scale=0.5]{LL_states}

\begin{itemize}

\item Advertising: il dispositivo rende noto agli ascoltatori della propria presenza, indicando la disponibilità ad una connessione ed inviando alcune informazioni utili.

\item Scanning: il dispositivo è in ascolto di tutte le informazioni trasmesse sui canali di advertise.

\item Initiating: un dispositivo in ascolto individua un Advertise di suo interesse ed indica la sua volontà di connettersi.

\item Connection: due o più dispositivi sono connessi.

\item Standby: il dispositivo è in uno stato di attesa caratterizzato da un basso consumo energetico.

\end{itemize}

Lo stato di Scanning può essere attivo (richiede informazioni aggiuntive) o passivo; Lo stato Connection anch'esso si divide in 2 sottostati: Central e Peripheral.

\item Logical Link Control and Adaptation Protocol (L2CAP): fornisce servizi sui dati ai livelli superiori, come ad esempio il Security Manager Protocol o l'Attribute Protocol. \'E responsabile inoltre della segmentazione e ricostruzione dei pacchetti da e verso i livelli inferiori.

\item Security Manager (SM): è responsabile del pairing dei dispositivi e della distribuzione delle chiavi crittografiche; BLE utilizza lo standard AES-128 bit per la cifratura dei dati ed il sistema di pairing per la distribuzione delle chiavi.

\item Attribute Protocol (ATT): gestisce la distribuzione delle informazioni riguardanti le coppie attributo-valore presenti in un device peripheral (Server)

\item Generic Access Profile (GAP): controlla \emph{advertising} e \emph{connection} . Divide i dispositivi bluetooth low energy in:
\begin{itemize}
\item Peripheral: normalmente dispositivi dotati di una limitata capacità di calcolo, come ad esempio un sensore di temperatura.
\item Central: dispositivo che gestisce la rete bluetooth e che richiede una maggiore capacità di calcolo; mentre un dispositivo periferico può avere una connessione con un solo dispositivo centrale, un centrale può gestire più dispositivi periferici andando a creare una Piconet\footnote{Rete Bluetooth composta da massimo otto dispositivi in relazione master-slave e fino a 255 dispositivi in modalità inattiva o parcheggio.}
\end{itemize} 

\item Generic Attribute Profile (GATT): entra in gioco a connessione avvenuta, definisce il modo in cui 2 dispositivi BLE scambiano dati utilizzando i concetti di \emph{Services} e \emph{Characteristics}.

\begin{itemize}
\item GATT server: è un peripheral device, che tramite il protocollo ATT permette al central di conoscere i dati che ha memorizzato in strutture di tipo service-characteristic.
\item GATT client: è il central device, che gestisce le comunicazioni e che richiede ai server periferici i dati raccolti.
\end{itemize}

\end{itemize}



 

%!TEX root = Tesi.tex

\section{RedBear Nano2}
Il dispositivo usato per implementare la funzione di sniffer è il Nano2 della società RedBear. Non più grande di una moneta (10mm x 18mm), ha un costo che si aggira attorno ai 30\$ .

\begin{figure}[H]
\includegraphics[width=300pt]{nano2_money}
\centering
\caption{RedBear Nano2, comparato con una moneta da 1/4 di dollaro.}
\end{figure}

Il nome in codice del modello è \lq MB-N2\rq ed integra al suo interno il chip della Nordic nRF52832, un'antenna integrata ed un totale di 32 pin di I/O\footnote{Input, Output: utilizzabili a piacimento come periferiche di ingresso o uscita} che offrono i servizi di UART, SPI, ADC e PWM.

\begin{figure}[H]
\includegraphics[width=250pt]{nano2_io}
\centering
\caption{Modulo SoC Nordic nRF52832 integrato nel RedBear Nano2.}
\end{figure}

\begin{samepage}
Il chip della Nordic Semiconductor nRF52832 Soc ha le seguenti caratteristiche
\begin{itemize}
\item[-] Processore ARM Cortex-M4F a 32 bit e 64 MHz.
\item[-] Bluetooth 4.2 certificato e compatibile con le specifiche 5.0 .
\item[-] NFC.
\item[-] 64 KB di ram.
\item[-] 512 KB di memoria Flash.
\item[-] FPU, unità di calcolo in virgola mobile.
\item[-] DSP, processore di segnale digitale.
\end{itemize}
\end{samepage}

Non disponendo di un'interfaccia standard per la comunicazione con un PC, il Nano2 necessita di una board di supporto sia per essere programmato, sia per comunicare informazioni a quest'ultimo.
La board che viene fornita assieme al Nano2 è chiamata DAPLink e disponibile nella versione 1.5 . Essa monta un processore ARM Cortex-M3 MCU ed è utilizzata oltre che per la programmazione, anche per il debug dei progetti su Nano2

\begin{figure}[H]
\includegraphics[width=330pt]{daplink}
\centering
\caption{Board DAPLink, un progetto open source del team ARM mbed.}
\end{figure}




%!TEX root = Tesi.tex

\section{Ambiente di sviluppo}
\subsection{Introduzione}
\'E stato scelto di sviluppare l'intero progetto di tesi in ambiente UNIX, usando come sistema operativo Linux, nello specifico la distribuzione di Ubuntu 17.04 a 64 bit denominata Zesty Zapus. Questa decisione è stata presa per integrare al meglio i progetti ed il codice già esistente e per creare una base di sviluppo futura che sia Open Source\footnote{Qualità di un sistema che consiste nell'essere di libero uso e riutilizzabile senza alcun costo, per ampliarne funzionalità e caratteristiche} e liberamente utilizzabile senza alcun vincolo di licenza. Non è stata comunque una decisione immediata perché il RedBear Nano2 integra un Soc\footnote{Sistem on a Chip, è un particolare circuito integrato che in un solo chip contiene un sistema completo} progettato dalla società Nordic Semiconductor, chiamato \lq\lq nRF52832\rq\rq, la quale mette a disposizione vari Tool, funzionalità e IDE di sviluppo per questo chip ideati e creati  per funzionare unicamente in ambiente Windows; fortunatamente esistono vari progetti e guide abbastanza dettagliate reperibili online, alcune delle quali supportate direttamente dalla Nordic stessa, che aiutano a settare ed integrare il materiale già esistente anche in ambiente UNIX.
Rimangono comunque delle forti limitazioni nello sviluppo degli applicativi per Nano2 su Linux, primo fra tutti la mancanza della possibilità di un ambiente di Debug che permetta l'analisi istruzione per istruzione del codice in esecuzione sul dispositivo, con la relativa possibilità di visualizzazione dei valori delle varie variabili durante l'esecuzione; questa limitazione è stata in parte sopperita utilizzando l'interfaccia UART, utilizzandola per inviare stringhe contenenti informazioni su quali parti del codice fossero o meno state realmente eseguite ed il valore di alcune variabili di interesse nelle varie fasi dell'esecuzione dell'applicativo sviluppato. 
Questa incompatibilità ha rallentato il processo di sviluppo su Linux ma non lo ha reso impossibile, ne tanto meno ne ha limitato le potenzialità.

\subsection{Installazione GNU toolchain}

\subsection{Configurazione SDK}

\subsection{Installazione IDE}


%!TEX root = Tesi.tex

\section{Progettazione sniffer}
Il primo passo per lo sviluppo è stato prendere confidenza sia con l'ambiente di lavoro che con il dispositivo della RedBear. Testare un semplice progetto, come il lampeggio di un led, si è comunque rivelata un'operazione non banale.
L'SDK che la Nordic fornisce contiene esempi modificati appositamente per le sue board di sviluppo, come la PCA10040 o la PCA10056, descritte nel capitolo \ref{nordic_board}, che come già accennato hanno una piedinatura e una dotazione di periferiche differente rispetto al nano2. Il primo passo è stato quindi quello di utilizzare una board supportata e modificarne la piedinatura delle periferiche connesse; il numero di led è stato ridotto a 1 solo connesso al pin 11 e sono stati rimossi i riferimenti a pulsanti connessi, perché non presenti sulla nostra periferica. \'E stato necessario modificare anche i collegamenti dei 4 pin relativi alla comunicazione UART come segue:
\begin{itemize}
\item[-] CTS pin 28
\item[-] RTS pin 2
\item[-] TX pin 29
\item[-] RX pin 30
\end{itemize}
Tutte queste modifiche sono state apportate al file pca10040.h che si trova nell'sdk nella cartella /components/boards .

Eseguendo una nuova compilazione e provando a scrivere il file .hex generato continuava a non funzionare. Dopo varie prove si è capito che mancavano delle librerie da cui il progetto dipendeva; esse sono contenute nel SOFTDEVICE, un file .hex che viene fornito già compilato nell'SDK, nella cartella /components/softdevice/s132/hex. Per tutti i progetti creati si usa la versione s132 del softdevice che è la più completa e contiene tutte le librerie necessarie per far funzionare il nano2 sia come dispositivo di tipo central che peripheral.

\subsection{UART}
Per visualizzare i dati intercettati via etere è necessario poter comunicare informazioni ad un dispositivo dotato di output video; nel nostro caso è stato utilizzato un PC, che avendo una gran capacità di memorizzazione può contenere tutti i dati catturati, per essere visualizzati ed analizzati anche in un secondo momento. Per trasferire questi dati abbiamo utilizzato una funzionalità supportata dal nano2, la trasmissione usando UART. \'E l'acronimo di Universal Asynchronous Receiver Transmitter, ovvero ricevitore trasmettitore asincrono/seriale, è un dispositivo hardware che converte flussi da un formato parallelo, tipicamente quelli usati all'interno di un processore, in formato seriale asincrono.
Inviare dati tramire UART è un'operazione che richiede pochi settaggi e quindi poche righe di codice, utilizzando le apposite funzioni di libreria messe a disposizione nella SDK. 

\begin{samepage}
All'interno di un progetto, va prima inizializzato il modulo fornendo le impostazioni che desideriamo usare:
\begin{itemize}
\item[-] Baudrate: il numero di simboli che vengono trasmessi in un secondo, determina la velocità di trasmissione.
\item[-] Control Flow: indica se utilizzare o meno le due linee aggiuntive destinate alla comunicazione UART ovvero, CTS (Clear To Send) e RTS (Ready To Send) per gestire lo scambio di dati, utile per annullare i conflitti trasmissivi.
\end{itemize}
Normalmente i settaggi sono inseriti nella funzione \emph{uart\_init()} che viene eseguita all'avvio del dispositivo.
\end{samepage}

Vanno inseriti dei caratteri di controllo per differenziare la fine di un pacchetto con l'inizio del successivo, ed eventualmente indicare informazioni aggiuntive al solo pacchetto, come la lunghezza dello stesso. \'E quindi stato inserito il carattere denominato MARKER del valore di 0xE0 all'inizio di ogni dato inserito; i 2 Byte successivi indicano la lunghezza dell'intero pacchetto, quindi comprendono anche i caratteri di controllo. Il quarto Byte viene usato per indicare il canale il cui è stato catturato il pacchetto e dal 5° Byte in poi si avrà ciò che è stato catturato, così come viene rilevato dal Nano2.
\begin{figure}[H]
\includegraphics[width=320pt]{uart_packet}
\centering
\caption{Pacchetto inviato tramite UART che contiene il pacchetto sniffato e i campi di controllo.}
\end{figure}
Nel caso in cui all'interno del pacchetto stesso ci siano Byte del valore di 0xE0, che hanno lo stesso valore del Marker, per evitare di confonderli si raddoppia il Byte. In questo modo se nel pacchetto ricevuto si incontrerà un Byte singolo del valore del Marker, allora quello sarà un vero Marker che indica l'inizio di una nuova trasmissione; negli altri casi sarà un Byte del pacchetto, si dovrà quindi scartare il Byte successivo, che non fa parte dei dati catturati dallo Sniffer.

\subsection{NRF RADIO}
Il cuore del progetto, ovvero l'intercettazione dei pacchetti, viene creato utilizzando la struttura NRF\_ RADIO, che contiene tutti i possibili campi settabili per far funzionare il dispositivo Radio, contenuto all'interno del SoC nRF52832, nelle modalità che si desiderano.
\begin{figure}[H]
\includegraphics[width=400pt]{radio_block_diagram}
\centering
\caption{Diagramma a Blocchi del trasmettitore radio.}
\end{figure}
Il dispositivo Radio contiene un ricevitore e trasmettitore a 2.4 GHz, un assemblatore e disassemblatore di pacchetti automatico, un generatore e verificatore di CRC automatico che permettono in modo facile e veloce di ottenere il pacchetto ascoltato partendo da ciò che è stato ricevuto via radio. Include inoltre un Device Address Match, utile per quelle funzionalità che richiedono il filtraggio dei pacchetti in automatico, per ricevere solo quelli destinati o provenienti da un determinato indirizzo. Si è dimostrato molto utile anche il sistema RSSI\footnote{Received Signal Strength Indicator}, ovvero un sistema che calcola l'intensità del segnale ricevuto, per essere eventualmente filtrato se troppo lontano o debole.
In ogni ambiente in cui si è andato ad attivare lo sniffer erano sempre presenti nell'etere pacchetti di altri dispositivi che andavano a sommarsi ai dispositivi usati per il test, rendendo più difficile l'individuazione a video dei pacchetti di interesse; tramite la ricezione della potenza del segnale è stato possibile escludere dall'invio e quindi dalla visualizzazione tutti quei pacchetti dei dispositivi lontani dallo sniffer, quindi di non interesse, garantendo la visualizzazione dei soli pacchetti provenienti dai dispositivi sotto test.
Il sistema Radio integrato nel SoC della Nordic è pensato per essere utilizzato con più protocolli trasmissivi, oltre al Bluetooth Low Energy; vanno quindi indicati, in fase di configurazione la lunghezza in Byte delle varie parti di un pacchetto BLE, con particolare specifica del campo Length del pacchetto. I campi di cui andare a settare la grandezza sono SO, LENGTH ed S1.
Tramite il valore di PCNF0 della struttura NRF\_ RADIO è possibile configurare la lunghezza dei campi suddetti.



\begin{minted}[fontsize=\footnotesize]{c}
NRF_RADIO->PCNF0 = (
	(((1UL) <<RADIO_PCNF0_S0LEN_Pos)& RADIO_PCNF0_S0LEN_Msk) |  
	(((2UL) <<RADIO_PCNF0_S1LEN_Pos)& RADIO_PCNF0_S1LEN_Msk) |  
	(((6UL) <<RADIO_PCNF0_LFLEN_Pos)& RADIO_PCNF0_LFLEN_Msk)   
);
\end{minted}

Dove la lunghezza di S0 è espressa in Byte e contiene il primo Byte dell'header, dove sono contenute le informazioni sul tipo di pacchetto e sul tipo degli eventuali indirizzi contenuti nel Payload (pubblici o privati.
La lunghezza di S1 è espressa invece in bit; nel nostro caso è uguale a 2 perché per i pacchetti di tipo advertise nel secondo Byte dell'header 2 bit sono RFU e attribuiti quindi a questo campo:
La lunghezza del campo length è espressa sempre in bit, 6 nel nostro caso e contiene appunto la lunghezza in Byte del payload.

Il valore del campo PCNF1 configura altri parametri, come la lunghezza massima che può assumere il Payload, eventuali Byte di espansione di quest'ultimo, lunghezza dell'indirizzo, il tipo di endian da utilizzare (introdotto nel capitolo \ref{endian_bit}) 

\begin{minted}[fontsize=\footnotesize,breaklines]{c}
NRF_RADIO->PCNF1 = (
	(((37UL) << RADIO_PCNF1_MAXLEN_Pos) & RADIO_PCNF1_MAXLEN_Msk)  |                     
	(((0UL) << RADIO_PCNF1_STATLEN_Pos) & RADIO_PCNF1_STATLEN_Msk) |                     
	(((3UL) << RADIO_PCNF1_BALEN_Pos) & RADIO_PCNF1_BALEN_Msk)     |                     
	(((RADIO_PCNF1_ENDIAN_Little) << RADIO_PCNF1_ENDIAN_Pos&RADIO_PCNF1_ENDIAN_Msk) | 
	(((1UL) << RADIO_PCNF1_WHITEEN_Pos) & RADIO_PCNF1_WHITEEN_Msk)                        
);
\end{minted}

Da configurare all'interno della struttura NRF\_ RADIO è anche il canale in cui essere in ascolto; esso influenza 2 campi:
\begin{minted}[fontsize=\footnotesize,breaklines]{c}
NRF_RADIO->FREQUENCY = frequency_resolver(); 
NRF_RADIO->DATAWHITEIV = channel;
\end{minted}

FREQUENCY è il campo in cui scrivere la frequenza di ascolto, indicata a scarti di 2 MHz e con prima frequenza utile il valore 0, ha quindi un campo di escursione da 0 a 80 MHz. Viene calcolata tramite la funzione \emph{frequency\_resolver()} che si basa sull'indice del canale desiderato.
DTAWHITEIV va settato semplicemente con l'indice del canale e serve per inizializzare l'algoritmo che andrà a rimuovere il whitening dai pacchetti ricevuti.

Tramite l'Access Address e il CRCInit che vengono forniti alla Radio si riceveranno solo i pacchetti che corrispondono a questi valori. Per i pacchetti di Advertising l'Access Address è fisso e vale 0x8E89BED6. Esso all'interno della struttura Radio è diviso in 2 parti: il PREFIX0 e BASE0; per impostare la Radio per la ricezione dei pacchetti advertise, l'indirizzo va diviso come segue:

\begin{minted}[fontsize=\footnotesize,breaklines]{c}
NRF_RADIO->PREFIX0 = 0x8e;
NRF_RADIO->BASE0 = 0x89bed600; 
\end{minted}
Per i pacchetti di una connessione l'AA varia, ed è deciso dal device Central al momento della connessione ed inviato al device Peripheral nella CONNECT\_ REQ.

I parametri da impostare per quanto riguarda la gestione del CRC sono il CRCCNF che indica il numero di Byte di lunghezza del CRC, il CRCINIT che per i pacchetti di Advertise ha un valore fisso a 0x555555 ma che per una connessione varia in accordo con il valore contenuto nella CONNECT\_ REQ, ed il CRCPOLY, che per tutte le trasmissioni BLE è fisso al valore 0x65B.

\begin{minted}[fontsize=\footnotesize,breaklines]{c}
NRF_RADIO->CRCCNF  = (RADIO_CRCCNF_LEN_Three << RADIO_CRCCNF_LEN_Pos) |
            (RADIO_CRCCNF_SKIPADDR_Skip << RADIO_CRCCNF_SKIPADDR_Pos); 
NRF_RADIO->CRCINIT = 0x555555;  
NRF_RADIO->CRCPOLY = 0x00065B;
\end{minted}

Dopo aver correttamente impostato la Radio è possibile mettere in ascolto il dispositivo per la ricezione dei pacchetti. la ricezione viene descritta ed attuata tramite una macchina a stati; il dispositivo Radio viene impostato e assume determinati stati, e controllandoli è possibile conoscere quando è stato ricevuto il pacchetto. Il diagramma degli stati per la ricezione e l'invio di pacchetti è mostrato in figura \ref{txrx_states}.

\begin{figure}[H]
\includegraphics[width=400pt]{txrx_states}
\centering
\caption{Diagramma degli stati per il dispositivo Radio che comprende le fasi di Invio e Ricezione.}
\label{txrx_states}
\end{figure}

Avviando la task per la ricezione, tramite il comando 
\begin{minted}[fontsize=\footnotesize,breaklines]{c}
NRF_RADIO-> TASKS_RXEN = 1U; 
\end{minted}
si imposta la Radio nello stato RXRU. Successivamente controllando quando il valore del campo EVENTS\_ READY assume il valore 1, si può sapere se il dispositivo Radio è pronto per la ricezione. Avviando poi il task di Start, la radio andrà nello stato RX. Per conoscere quando un pacchetto è stato completamente ricevuto si deve controllare il valore del campo EVENTS\_ END, se è 1 un pacchetto è stato ricevuto.
Il controllo della correttezza del CRC non avviene in automatico ma va verificato controllando il valore del campo CRCSTATUS, se 1 il CRC calcolato corrisponde a quello inviato con il pacchetto e quindi il pacchetto è stato ricevuto senza errori. Ad ogni ricezione la Radio va disabilitata, per poter tornare allo stato iniziale e ricevere il pacchetto successivo.

Ogni pacchetto ricevuto deve essere inviato alla UART; vanno prima aggiunti i caratteri di controllo e il separatore (Marker) dei pacchetti, ed è necessario controllare se il pacchetto catturato contenga il valore 0xE0, in quel caso va aggiunto un ulteriore Byte al pacchetto in successione a quest'ultimo per evitare che venga riconosciuto erroneamente come un Marker. Va quindi calcolata la lunghezza totale del pacchetto così modificato e inserita nei Byte 1 e 2  cosicché il ricevitore, dall'altro lato della comunicazione seriale, sappia quanto è lungo il pacchetto e riesca ad identificare eventuali errori di trasmissione se si vede arrivare un Marker prima che il pacchetto inviato abbia raggiunto la lunghezza dichiarata.
Sul dispositivo Nano2 esiste una funzione di libreria che permette di inviare un singolo Byte alla UART e conoscerne il momento dell'effettivo invio, così da evitare di terminare l'invio precedente e sovrascriverlo con uno nuovo; la funzione va quindi chiamata con un while che ne attende il completamento dell'invio:

\begin{minted}[fontsize=\footnotesize,breaklines]{text}
while(NRF_SUCCESS != app_uart_put(ch)); 
\end{minted}

\subsection{Ricevitore seriale}
Dall'altro lato della comunicazione UART è stato usato un PC con sistema operativo Linux, nello specifico Ubuntu 17.04, che ha il vantaggio di supportare nativamente le trasmissioni in seriale andando a creare uno speciale file, /dev/ttyACM0 , in cui si trovano scritti tutti i dati ricevuti da seriale. Leggendo il file è si ottengono i pacchetti inviati dal Nano2 che vanno poi elaborati, visualizzati a video e salvati su un file per analisi future.
\'E stato creato un programma, scritto in linguaggio C, che permette di svolgere le funzioni sopra citate; essendo la trasmissione asincrona, esso deve innanzitutto aspettare la ricezione di un Marker singolo, che indica l'inizio di un nuovo pacchetto, andare a leggere i 2 Byte successivi e ottenerne la lunghezza, e mediante un ciclo for, andare a leggere tutti i Byte successivi e presentarli a video, salvandoli anche sempre su un file secondario.
Per svolgere queste funzioni è stata progettata una macchina a stati, replicata poi in linguaggio C mediante il costrutto \emph{Switch(state)}, mostrata in figura \ref{uart_machine_state}.

\begin{figure}[H]
\includegraphics[width=400pt]{uart_machine_state}
\centering
\caption{Diagramma della macchina a stati implementata in C per ricevere i dati dalla UART.}
\label{uart_machine_state}
\end{figure}

Lo stato 0 analizza tutti i Byte scritti nel file alla ricerca del Marker singolo, se lo trova avanza nello stato 1. Qui se il Byte successivo è anch'esso un Marker, siamo di fronte ad un falso Marker e si deve tornare nello stato 0 ad aspettare un nuovo Maker, se invece il Byte è diverso allora esso è il byte più significativo della lunghezza, che va letto e salvato in una variabile; si passa automaticamente nello stato 2 dove si legge il restante Byte della lunghezza avendo quindi ad ora la lunghezza del pacchetto da stampare. Nello stato 3 si va a leggere il 4° Byte del pacchetto che indica in che canale è stato catturato e si va a scrivere subito l'informazione; si procede direttamente nello stato 4 dove si andranno a leggere e stampare tutti i restanti Byte del pacchetto, andando di volta in volta a decrementare la variabile length finché non assume il valore 0 ed il pacchetto è stato interamente letto e si torna quindi nello stato iniziale 0.

\subsubsection{Seguire una connessione}
Per catturare i pacchetti di tipo Advertise è sufficiente sintonizzarsi su un canale di Advertise (37, 38 o 39) e tramite l'AA e CRCInit noti per tutti i pacchetti di Advertise si ricevono tutte le comunicazioni dei dispositivi che vogliono farsi individuare nell'area adiacente allo Sniffer. Un qualsiasi dispositivo che invia pacchetti di Advertise lo fa sempre sui i 3 canali dedicati; uno sniffer quindi è sufficiente che si metta in ascolto su uno dei 3 per poter catturare tutti i pacchetti di questo tipo. Il procedimento si fa più complicato quando si deve andare ad intercettare il pacchetto di CONNECT\_ REQ, perchè viene inviato soltanto su 1 dei 3 canali, ed inviato una sola volta.
Per ovviare a questo problema ci sono varie soluzioni; la prima, la più elementare ma che si adatta a tutti i possibili scenari, è fissarsi su un canale e forzare il dispositivo target a riconnettersi più volte finché non manderà la CONNECT\_ REQ sul canale in cui siamo in ascolto. Una seconda soluzione è creare un dispositivo di Advertising a piacimento che faccia Advertise solo sul canale in cui siamo in ascolto; il dispositivo che vorrà connettersi con quest'ultimo dovrà obbligatoriamente, seguendo le specifiche del protocollo BLE, inviare la sua richiesta di connessione sul canale in cui siamo in ascolto. L'ultima soluzione, la più efficacie ma anche più complessa e costosa, è avere 3 sniffer, uno per ogni canale di Advertise che forniscono una sicurezza del 100\% di riuscire ad intercettare il pacchetto di CONNECT\_ REQ.
Per seguire successivamente la connessione bisognerà ottenere alcuni valori dal pacchetto di connessione catturato; l'AA contenuto nei Byte [15],[16],[17],[18] CRCInit nei Byte [19],[20],[21] e il valore dell'Hop Increment contenuto negli ultimi 5 bit del pacchetto. Successivamente si deve impostare la struttura NRF\_ RADIO con i valori estratti dal pacchetto di connessione per catturare unicamente i pacchetti della connessione;
La parte più complessa è gestire il cambio di canale con le tempistiche esatte per riuscire a ricevere tutti i pacchetti della connessione. Dalle specifiche Low Energy è noto che l'intervallo di connessione, ovvero il tempo in cui la connessione avviene entro un determinato canale, è fisso e deciso in fase di connessione. Questo intervallo inizia nello stesso istante in cui il dispositivo Central invia il primo pacchetto; l'istante viene definito come Anchor Point, ovvero punto di ancoraggio per la connessione.
Non esiste però una funzione di libreria della Nordic che permette di ottenere questo istante, ci si può però basare sull'istante in cui viene ricevuto l'indirizzo, che si può conoscere tramite funzioni di libreria,  e togliere il tempo che è trascorso per ottenere l'istante iniziale in cui il pacchetto è stato trasmesso, dato che l'indirizzo viene trasmesso sempre un numero fissato di Byte dopo l'inizio del pacchetto e la velocità di trasmissione è nota e costante.
Dopo aver fissato l'anchor point, utilizzando la gestione del channel hopping descritta nel capitolo \ref{freqHopping}, è possibile seguire la connessione ed ottenerne tutti i pacchetti.

\subsection{Problematiche riscontrate}
Il progetto dello Sniffer che riuscisse anche a seguire una connessione tra due dispositivi target ha presentato non pochi scogli; la prima grossa problematica è nata sulla cattura della CONNECT\_ REQ. Questo pacchetto, fondamentale per poter ascoltare una connessione, nelle prime versioni dello Sniffer non riusciva ad essere mai catturato. Indipendentemente dai canale di Advertise in cui si era in ascolto o dal numero di tentativi di connessione che di provavano o dalla potenza trasmissiva usata, il pacchetto di connessione non compariva mai nell'elenco dei pacchetti catturati.
La soluzione di questo problema è stata trovata analizzando con il dispositivo USRP i tempi che intercorrevano tra i vari pacchetti di Advertise e la richiesta di connessione; si è notato che essa avveniva con un tempo minore del millisecondo dopo un pacchetto di Advertise. Il problema è stato individuato nella lentezza della trasmissione del dispositivo UART che impegnava per più di un millisecondo la CPU e che quindi non riusciva a riattivare in tempo il dispositivo Radio per ricevere la CONNECT\_ REQ. Inibendo l'invio dei pacchetti di Advertise catturati, ma inviando solamente quello relativo all'evento di connessione, è stata immediata la sua cattura.

Successivamente alla riuscita della cattura del pacchetto di connessione si è quindi provato a impostarci su un canale di connessione, in modo fisso, per vedere se si riusciva nella cattura di un pacchetto.
Partendo dal presupposto che i valori dell'AA e del CRCInit che il Nano2 catturava non erano da usare nell'ordine in cui li inseriva nel pacchetto catturato ed usando come similitudine il comportamento che utilizzava per altri campi del pacchetto, si è pensato che il loro valore reale fosse una specularità Byte a Byte del valore reperibile dal pacchetto. Seppur con svariati tentativi di connessione e inibendo anche la verifica del CRC, concentrandosi quindi unicamente sulla correttezza dell'Access Address, non si è riuscito a catturare nessun pacchetto nei canali data. \'E stato grazie ad un uso verboso dei log che si è individuato il problema: il dispositivo poco dopo l'avvio andava in crash. Questo comportava il riavvio dello stesso con la conseguenza che tutti i dati estratti dal pacchetto di connessione andavano sempre persi e quindi era impossibile catturare un qualsiasi pacchetto nei canali data. Solo risolvendo la causa del crash, ovvero un Null Pointer sulla struttura UART, è stato possibile catturare il primo pacchetto nei canali data.

\section{Sviluppi futuri}
Con lo svolgimento di questo lavoro di Tesi si sono create le basi per la creazione di un dispositivo di Relay, un sistema in grado di catturare tutti i pacchetti inviati da un certo dispositivo e replicarli ad una distanza di molto maggiore della portata trasmissiva del protocollo BLE; questo sistema può venire usato per implementare l'attacco Man in the Middle, ovvero un attacco informatico in cui un dispositivo attaccante ritrasmette e/o altera la comunicazione tra due parti che credono di comunicare direttamente tra di loro; nell'ambito Bluetooth questo attacco permette di far credere a 2 dispositivi lontani  di essere a breve distanza e di comunicare. 
La riuscita di questo tipo di attacco permette di svelare le vulnerabilità di tutti quei sistemi che utilizzano la vicinanza tra 2 dispositivi Bluetooth come metodo di accesso a dispositivi protetti da un sistema di sblocco, normalmente con password.
Ad oggi sono sempre di più le applicazioni che utilizzano un sistema keyless per permettere l'accesso ai soli autorizzati, basti pensare ad una serratura della porta principale di un'abitazione, che viene automaticamente sbloccata quando qualcuno in possesso della chiave si trova nelle strette vicinanze della porta e apre semplicemente la maniglia; con un relay di questo tipo un potenziale malintenzionato potrebbe intercettare il proprietario della casa mentre è ad esempio al supermercato, avvicinarlo con un dispositivo di ascolto mentre un suo complice si trova davanti alla sua porta di casa e con il dispositivo di ritrasmissione aprire facilmente la porta di casa.
L'utilizzo di un dispositivo a basso costo rende più facile ed economica la creazione di un relay bluetooth, che può essere reperito ed utilizzato da un numero maggiore di attaccanti; come citato nelle premesse è molto importante che il protocollo trasmissivo BLE sia sicuro ed adotti misure per impedire che attacchi di questo tipo possano funzionare.

Per la creazione di un relay funzionante deve ancora essere sviluppata la parte di trasmissione dei pacchetti catturati a lunga distanza, attraverso mezzi trasmissivi come la rete cellulare utilizzando il protocollo LTE. Per la creazione del dispositivo che ritrasmette i pacchetti catturati, è già stato creato del codice funzionante che permette la trasmissione di pacchetti completamente personalizzati, utilizzato ampiamente nelle fasi di test dello Sniffer.

%!TEX root = Tesi.tex

\section{Bibliografia}

\begin{thebibliography}{9}

\bibitem{bt_core_v4} 
Bluetooth SIG Working Groups. 
\textit{Bluetooth\textsuperscript{\textregistered} Core Specification v4.0}
\mbox{30 June 2010}.

\bibitem{adafruitweb} 
Introduction to Bluetooth Low Energy, 
\\\texttt{https://learn.adafruit.com/introduction-to-bluetooth-low-energy}
 
\bibitem{dmitryweb} 
CRC and data whitening, 
\\\texttt{http://dmitry.gr/index.php?r=05.Projects\&proj=11.\% 20Bluetooth\%20LE\%20fakery}

\bibitem{eetimesweb} 
Bluetooth LE Stack Partition, 
\\\texttt{https://www.eetimes.com/document.asp?doc\_id=1278966}

\bibitem{armweb} 
Bluetooth LE Stack Partition, 
\\\texttt{https://developer.arm.com/open-source/gnu-toolchain/gnu-rm/downloads}

\bibitem{sdkweb} 
Bluetooth LE Stack Partition, 
\\\texttt{http://developer.nordicsemi.com/nRF5\_SDK/nRF5\_SDK\_v14.x.x}

\bibitem{gnueclipseweb} 
GNU MCU Eclipse IDE for C/C++ Developers Neon, 
\\\texttt{https://github.com/gnu-mcu-eclipse/org.eclipse.epp.packages/releases/}


\end{thebibliography}

\end{document}\grid
\grid
