%!TEX root = Tesi.tex

\section{Bluetooth Low Energy}
\subsection{Introduzione}
Il Bluetooth Low Energy (BLE) è una tecnologia wireless introdotta nello standard 4.0, disegnata e commercializzata dal SIG\footnote{Bluetooth Special Interest Group: organizzazione che regolamenta e definisce gli standard  per la trasmissione dati tramite la tecnologia Bluetooth.}; ideata per trasmissioni di dati di breve dimensione, essa si differenzia dal Bluetooth classico per un inferiore consumo di energia, mantenendo la stessa portata trasmissiva.

\subsection{Banda Trasmissiva}
BLE opera alla stessa banda di frequenza del bluetooth classico, 2.400 - 2.4835 GHz, ma utilizza un numero inferiore di canali, 40 canali da 2 MHz l'uno.
Nel canale trasmissivo i dati sono trasmessi con una modulazione GFSK.\footnote{Gaussin Frequency Shift Keying, limita la larghezza dello spettro trasmissivo tramite un filtro di tipo Gaussiano.}
I bitrate trasmissivi sono limitati ai valori 125 kbit/s, 1 Mbit/s o 2 Mbit/s.

Bluetooth Low Energy utilizza il \emph{Frequency Hopping}, una tecnica trasmissiva che consiste nel cambiare canale trasmissivo per ogni pacchetto inviato durante una trasmissione; l'utilizzo di questa tecnica permette di ridurre i problemi di interferenza con gli altri canali ed allo stesso tempo di aumentare la sicurezza delle trasmissioni.

\subsection{Stack protocollare}
Storicamente lo stack Bluetooth è diviso in 2 componenti fondamentali: il \emph{Controller} e l'\emph{Host}.
La parte di Controller è quella che si occupa di eseguire le componenti di basso livello dello stack necessarie per gestire i pacchetti scambiati a livello fisico e le loro temporizzazioni; la parte di Host comprende le componenti di alto livello tra cui profili e API\footnote{Application Program Interface: insieme di procedure utili a svolgere uno specifico compito}. La parte di Host, diversamente da quella di controllo, astrae dall'hardware ed è caratterizzata da una gestione meno rigida delle temporizzazioni.

L'\emph{HCI}\footnote{Host Controller Interface}, l'interfaccia tra Host e Controller si occupa di mettere in comunicazione queste due componenti fondamentali, utilizzando canali comunicativi quali UART o USB.
Se i due componenti sono montati su uno stesso chip, come nel caso dei SoC\footnote{System on a Chip: sistema completamente contenuto su un solo Chip}, allora l'interfaccia HCI è opzionale e può essere omessa.

\includegraphics[scale=0.5]{stack_ble.png}

\begin{itemize}
\item Physical Layer (PHY): è il livello fisico, ovvero l'etere; lavora nella banda di frequenza 2.4 GHz ISM\footnote{Industrial, Scientific and Medical: frequenze riservate alle applicazioni di radiocomunicazioni non commerciali, ovvero per uso industriale, scientifico e medico.} Si compone di 40 canali da 2 MHz ognuno suddivisi in 3 per advertising e 37 per lo scambio dati.

\item Link Layer (LL): si occupa della gestione della sequenza e della temporizzazione dei pacchetti scambiati. Dialoga con i nodi vicini e scambia informazioni quali i parametri di connessione e controllo di flusso.\linebreak 
\'E una macchina a stati che si compone di 5 stati: Standby, Advertising, Scanning, Initiating, Connection. 

\includegraphics[scale=0.5]{LL_states}

Lo stato di Scanning può essere attivo (richiede informazioni aggiuntive) o passivo; Lo stato Connection anch'esso si divide in 2 sottostati: Central e Peripheral.

\item Logical Link Control and Adaptation Protocol (L2CAP): fornisce servizi sui dati ai livelli superiori, come ad esempio il Security Manager Protocol o l'Attribute Protocol. \'E responsabile inoltre della segmentazione e ricostruzione dei pacchetti da e verso i livelli inferiori.

\item Security Manager (SM): è responsabile del pairing dei dispositivi e della distribuzione delle chiavi crittografiche; BLE utilizza lo standard AES-128 bit per la cifratura dei dati ed il sistema di pairing per la distribuzione delle chiavi.

\item Attribute Protocol (ATT): gestisce la distribuzione delle informazioni riguardanti le coppie attributo-valore presenti in un device peripheral (Server)

\item Generic Access Profile (GAP): 
\end{itemize}



