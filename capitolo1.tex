%!TEX root = Tesi.tex

\section{Bluetooth Low Energy}
\subsection{Introduzione}
Il Bluetooth Low Energy (BLE) è una tecnologia wireless introdotta nello standard 4.0, disegnata e commercializzata dal SIG\footnote{Bluetooth Special Interest Group: organizzazione che regolamenta e definisce gli standard  per la trasmissione dati tramite la tecnologia Bluetooth.}; ideata per trasmissioni di dati di breve dimensione, essa si differenzia dal Bluetooth classico per un inferiore consumo di energia, mantenendo la stessa portata trasmissiva.

\subsection{Banda Trasmissiva}
BLE opera alla stessa banda di frequenza del bluetooth classico, 2,400 - 2,4835 GHz, ma utilizza un numero inferiore di canali, 40 canali da 2 MHz l'uno.
Nel canale trasmissivo i dati sono trasmessi con una modulazione GFSK.\footnote{Gaussin Frequency Shift Keying, limita la larghezza dello spettro trasmissivo tramite un filtro di tipo Gaussiano.}
I bitrate trasmissivi sono limitati ai valori 125 kbit/s, 1 Mbit/s o 2 Mbit/s.

\subsubsection{Mapping tra canali e frequenze}
Il protocollo BLE utilizza 40 canali fisici differenti per la trasmissione, con una numerazione che va da 0 a 39. Ogni canale ha una frequenza centrata in $2402 + (k * 2)$ MHz, dove k assume il valore del canale considerato.
Questi 40 canali si dividono in:
\begin{itemize}
\item 3 canali 0, 12 e 39 centrati nelle frequenze 2402 MHz, 2426 MHz e 2480 MHz ed utilizzati per l'Advertise. La distribuzione di questi canali non è casuale: essi difatti sono i canali utilizzati da un dispositivo per far conoscere la propria presenza ai vicini, è quindi fondamentale che almeno uno di essi sia disponibile per la trasmissione, quindi privo di disturbi generati da altri sistemi che operano nella stessa banda trasmissiva. Sono stati distribuiti lontani tra di loro cercando di utilizzare tutta la banda disponibile per massimizzare la possibilità di avere un canale libero per la trasmissione. Un altro motivo della scelta è stato quello di poter facilitare la co-esistenza con la tecnologia Wifi, che utilizza la stessa banda per trasmettere, centrata principalmente nei canali Wifi 1, 6 e 11.

\begin{figure}[H]
\includegraphics[width=300pt]{channel_mapping_wifi}
\centering
\caption{sovrapposizione alla banda BLE dei 3 canali usati principalmente nella comunicazione wifi; si noti come i 3 canali di advertise non rientrino in nessuno di questi 3 canali.}
\end{figure}

\item i 37 canali restanti sono utilizzati per la trasmissione di dati che vengono utilizzati solo a connessione avvenuta per lo scambio di informazioni.
\end{itemize} 


Per una miglior leggibilità ed un più facile utilizzo, si è scelto di mappare i canali reali con dei corrispondenti indici fittizi che vedono nelle numerazioni da 0 a 36 i canali data e 37, 38, 39 per i canali di Advertise;
\begin{itemize}
\item[-] canale fisico 0, di Advertise a frequenza 2402 MHz, mappato come canale 37.
\item[-] canale fisico 12, di Advertise a frequenza 2426 MHz, mappato come canale 38.
\item[-] canale fisico 39, di Advertise a frequenza 2480 MHz, mappato come canale 39. 
\item[-] canale fisico 1, di scambio dati a frequenza 2404 MHz, mappato come canale 0.
\item[-] \dots
\item[-] canale fisico 38, di scambio dati a frequenza 2478 MHz, mappato come canale 36.
\end{itemize}

\begin{figure}[H]
\caption{Distribuzione dei canali nella banda trasmissiva BLE, con il riferimento all'indice del canale mappato e la relativa frequenza centrale.}
\includegraphics[width=350pt]{channel_mapping}
\centering
\label{channel_mapping}
\end{figure}




\subsubsection{Adaptive Frequency Hopping}
Bluetooth Low Energy utilizza il \emph{Frequency Hopping}, una tecnica trasmissiva che consiste nel cambiare canale trasmissivo per ogni pacchetto inviato durante una trasmissione; l'utilizzo di questa tecnica permette di ridurre i problemi di interferenza con gli altri canali ed allo stesso tempo di aumentare la sicurezza delle trasmissioni. Esso è definito adattivo perché permette, in fase di connessione, di andare ad escludere determinati canali perché considerati troppo disturbati. L'algoritmo che regola questo cambio di canali è determinato da una formula che calcola il successivo canale in cui trasmettere, avendo come parametro l'ultimo canale usato e il numero di canali da saltare

\[Ch = (Ch_{-1} + HopIncrement)mod 37\]

Dove:
\begin{itemize}
\item[] \emph{Ch}: è il prossimo canale in cui trasmettere.
\item[] $Ch_{-1}$: è l'ultimo canale in cui si è trasmesso.
\item[] \emph{HopIncrement} : costante che indica il numero di canali da saltare, che viene scambiato dai dispositivi in fase di connessione.
\end{itemize}

\subsection{Stack protocollare}
Storicamente lo stack Bluetooth è diviso in 2 componenti fondamentali: il \emph{Controller} e l'\emph{Host}.
La parte di Controller è quella che si occupa di eseguire le componenti di basso livello dello stack necessarie per gestire i pacchetti scambiati a livello fisico e le loro temporizzazioni; la parte di Host comprende le componenti di alto livello tra cui profili e API\footnote{Application Program Interface: insieme di procedure utili a svolgere uno specifico compito}. La parte di Host, diversamente da quella di controllo, astrae dall'hardware ed è caratterizzata da una gestione meno rigida delle temporizzazioni.

L'\emph{HCI}\footnote{Host Controller Interface}, l'interfaccia tra Host e Controller si occupa di mettere in comunicazione queste due componenti fondamentali, utilizzando canali comunicativi quali UART o USB.
Se i due componenti sono montati su uno stesso chip, come nel caso dei SoC\footnote{System on a Chip: sistema completamente contenuto su un solo Chip}, allora l'interfaccia HCI è opzionale e può essere omessa.

\begin{figure}[H]
\includegraphics[scale=0.5]{stack_ble.png}
\centering
\end{figure}

\begin{itemize}
\item Physical Layer (PHY): è il livello fisico, ovvero l'etere; lavora nella banda di frequenza 2,4 GHz ISM\footnote{Industrial, Scientific and Medical: frequenze riservate alle applicazioni di radiocomunicazioni non commerciali, ovvero per uso industriale, scientifico e medico.} Si compone di 40 canali da 2 MHz ognuno suddivisi in 3 per advertising e 37 per lo scambio dati.

\item Link Layer (LL): si occupa della gestione della sequenza e della temporizzazione dei pacchetti scambiati. Dialoga con i nodi vicini e scambia informazioni quali i parametri di connessione e controllo di flusso.\linebreak 
\'E una macchina a stati che si compone di 5 stati: Standby, Advertising, Scanning, Initiating, Connection. 

\begin{figure}[H]
\includegraphics[scale=0.5]{LL_states}
\centering
\end{figure}

\begin{itemize}

\item Advertising: il dispositivo rende noto agli ascoltatori della propria presenza, indicando la disponibilità ad una connessione ed inviando alcune informazioni utili.

\item Scanning: il dispositivo è in ascolto di tutte le informazioni trasmesse sui canali di advertise.

\item Initiating: un dispositivo in ascolto individua un Advertise di suo interesse ed indica la sua volontà di connettersi.

\item Connection: due o più dispositivi sono connessi.

\item Standby: il dispositivo è in uno stato di attesa caratterizzato da un basso consumo energetico.

\end{itemize}

Lo stato di Scanning può essere attivo (richiede informazioni aggiuntive) o passivo; Lo stato Connection anch'esso si divide in 2 sottostati: Central e Peripheral.

\item Logical Link Control and Adaptation Protocol (L2CAP): fornisce servizi sui dati ai livelli superiori, come ad esempio il Security Manager Protocol o l'Attribute Protocol. \'E responsabile inoltre della segmentazione e ricostruzione dei pacchetti da e verso i livelli inferiori.

\item Security Manager (SM): è responsabile del pairing dei dispositivi e della distribuzione delle chiavi crittografiche; BLE utilizza lo standard AES-128 bit per la cifratura dei dati ed il sistema di pairing per la distribuzione delle chiavi.

\item Attribute Protocol (ATT): gestisce la distribuzione delle informazioni riguardanti le coppie attributo-valore presenti in un device peripheral (Server)

\item Generic Access Profile (GAP): controlla \emph{advertising} e \emph{connection} . Divide i dispositivi bluetooth low energy in:
\begin{itemize}
\item Peripheral: normalmente dispositivi dotati di una limitata capacità di calcolo, come ad esempio un sensore di temperatura.
\item Central: dispositivo che gestisce la rete bluetooth e che richiede una maggiore capacità di calcolo; mentre un dispositivo periferico può avere una connessione con un solo dispositivo centrale, un centrale può gestire più dispositivi periferici andando a creare una Piconet\footnote{Rete Bluetooth composta da massimo otto dispositivi in relazione master-slave e fino a 255 dispositivi in modalità inattiva o parcheggio.}
\end{itemize} 

\item Generic Attribute Profile (GATT): entra in gioco a connessione avvenuta, definisce il modo in cui 2 dispositivi BLE scambiano dati utilizzando i concetti di \emph{Services} e \emph{Characteristics}.

\begin{itemize}
\item GATT server: è un peripheral device, che tramite il protocollo ATT permette al central di conoscere i dati che ha memorizzato in strutture di tipo service-characteristic.
\item GATT client: è il central device, che gestisce le comunicazioni e che richiede ai server periferici i dati raccolti.
\end{itemize}
\end{itemize}

\subsection{Formato dei pacchetti}

Il Link Layer (LL) ha un unico formato per tutti i pacchetti scambiati, che quindi è uguale sia per i pacchetti di tipo Advertise, sia per i pacchetti di scambio di dati.
In figura \ref{pacchetto_LL} è mostrato il formato dei pacchetti

\begin{figure}[H]
\includegraphics[width=340pt]{pacchetto_LL}
\centering
\caption{formato dei pacchetti Link Layer}
\label{pacchetto_LL}
\end{figure}

\subsubsection{Preamble}
Composto da 8 bit, è utile al ricevitore del pacchetto per sincronizzarsi sulla frequenza portante, e per impostare l'\emph{Automatic Gain Control} (AGC).
I pacchetti di Advertise hanno come preambolo $10101010b$. Per i pacchetti data il valore del preambolo è determinato dal LSB\footnote{Less Significant Bit, il bit meno significativo.} dell'Access Address; se è 1 il preambolo è $01010101b$, se è 0 il preambolo vale $10101010b$

\subsubsection{Access Address}
Campo composto da 4 Byte; per i pacchetti di Advertise assume un valore costante pari a 0x8E89BED6. Per le trasmissioni di dati l'Access Address deve essere diverso per ogni connessione; dovrebbe essere un valore casuale (con alcune restrizioni) di 32 bit generato dal dispositivo che inizia la connessione ed inviato al secondo dispositivo durante la connessione tramite il pacchetto di Connection Request. 

\subsubsection{PDU}
Composta da un numero variabile da 2 a 39 byte, è l'acronimo di Protocol Data Unit, ovvero l'unità di informazione scambiata. A seconda che si abbia un pacchetto di Advertise o un pacchetto dati, la PDU assume una struttura differente.

\subsubsection{CRC}
Alla fine di ogni pacchetto Link Layer c'è un CRC,Il cyclic redundancy check (ovvero controllo a ridondanza ciclica), di 24 bit. Esso serve per rilevare eventuali errori di trasmissione, quindi per sapere se il pacchetto ricevuto è corrotto.
Il CRC viene calcolato sul campo PDU, partendo dal LSB; il polinomio per il calcolo è:
\[x^24 + x^10 + x^9 + x^6 + x^4 + x^3 + x + 1\] 
Se il calcolo del CRC viene fatto su un PDU di tipo Advertise, deve essere inizializzato con il valore 0x555555;
Se viene fatto su un pacchetto dati, deve essere inizializzato ad un valore scambiato in fase di comunicazione (di 3 Byte), contenuto nella Connection Request.
In posizione 0 va inserito il bit meno significativo del valore di inizializzazione, ma il CRC viene inviato con il bit più significativo (bit 23) in posizione iniziale.

\subsection{Advertising PDU}
Il PDU di un pacchetto di Advertise si compone di un header di 16 bit ed un payload con una lunghezza variabile dai 6 ai 37 Byte.

\begin{figure}[H]
\includegraphics[width=300pt]{Advertising_pdu}
\centering
\caption{Advertising PDU}
\end{figure}

\noindent L'Header di questo PDU si compone di vari campi, come mostrato in figura \ref{Advertising_pdu_header}

\begin{figure}[H]
\includegraphics[width=300pt]{Advertising_pdu_header}
\centering
\caption{Advertising PDU Header}
\label{Advertising_pdu_header}
\end{figure}

Il campo PDU type verrà approfondito in seguito, assieme al significato dei campi TxAdd e RxAdd; Il campo length, di 6 bit quindi con un valore massimo di 63, ma settabile dalle specifiche ad un valore massimo di 37, indica i Byte di dimensione del Payload.


\subsubsection{PDU type}

\begin{figure}[H]
\includegraphics[width=200pt]{Advertising_pdu_type}
\centering
\caption{Codifiche dei possibili tipi di PDU di un pacchetto di Advertise}
\end{figure}
 
Campo composto da 4 bit; i vari tipi di PDU si raggruppano in 3 insiemi distinti: Advertising PDU, Scanning PDU e Initiating PDU.

\subsubsection{Advertising PDU}

Questi PDU sono inviati da un dispositivo in stato di Advertise e ricevuti da un sistema in stato di scanning.

\begin{itemize}
\item ADV\_ IND: evento di advertising che notifica la disponibilità ad una connessione indiretta, ovvero senza specifiche sull'indirizzo da cui ricevere la richiesta di connessione.
Il payload di questo tipo di PDU è mostrato in figura \ref{adv_ind_payload}, il campo TxAdd dell'header se settato a 1 indica che l'indirizzo del campo AdvA del payload è casuale, se settato a 0 esso è pubblico.

\begin{figure}[H]
\includegraphics[width=150pt]{adv_ind_payload}
\centering
\caption{Payload di un PDU di tipo ADV\_ IND }
\label{adv_ind_payload}
\end{figure}

Si compone dei campi AdvA di 6 byte, che indica l'indirizzo, pubblico o privato, dell'advertiser, ovvero di chi sta inviando il pacchetto, e il campo AdvData che contiene eventuali informazioni aggiuntive ed ha una lunghezza massima di 31 Byte.

\item ADV\_ DIRECT\_ IND: evento di connessione diretta, in cui nel payload, mostrato in figura \ref{adv_direct_ind_payload} è indicato l'indirizzo da cui ci si aspetta una connessione. Il campo TxAdd dell'header se settato a 1 indica che l'indirizzo del campo AdvA del payload è casuale, se settato a 0 esso è pubblico;
il campo RxAdd dell'header se settato a 1 indica che l'indirizzo del campo InitA del payload è casuale, se settato a 0 esso è pubblico. 

\begin{figure}[H]
\includegraphics[width=150pt]{adv_direct_ind_payload}
\centering
\caption{Payload di un PDU di tipo ADV\_ DIRECT\_ IND }
\label{adv_direct_ind_payload}
\end{figure}

Il campo AdvA contiene l'indirizzo dell'Advertiser, sia esso pubblico o privato, ed il campo InitA indica l'indirizzo da cui ci si aspetta la connessione. Il payload di questo tipo non contiene nessuna informazione aggiuntiva.

\item ADV\_ NONCONN\_ IND: evento di non connessione; utilizzato per applicazioni di tipo BEACON, ovvero per condividere informazioni a chiunque è in ascolto. Nel  payload, mostrato in figura \ref{adv_nonconn_ind_payload} è indicato l'indirizzo dell'Advertiser. Il campo TxAdd dell'header se settato a 1 indica che l'indirizzo del campo AdvA del payload è casuale, se settato a 0 esso è pubblico;

\begin{figure}[H]
\includegraphics[width=150pt]{adv_nonconn_ind_payload}
\centering
\caption{Payload di un PDU di tipo ADV\_ NONCONN\_ IND }
\label{adv_nonconn_ind_payload}
\end{figure}

Il campo AdvA contiene l'indirizzo dell'Advertiser, sia esso pubblico o privato, ed il campo AdvData che contiene le informazioni da condividere ed ha una lunghezza massima di 31 Byte.

\item ADV\_ SCAN\_ IND: evento di connessione indiretta in cui si informa che si hanno ulteriori informazioni da condividere; chi è interessato a queste informazioni manderà all'Advertiser un PDU di tipo SCAN\_ REQ, e l'advertiser risponderà con un PDU di tipo SCSN\_ RSP contenente le informazioni aggiuntive. Nel  payload, mostrato in figura \ref{adv_scan_ind_payload} è indicato l'indirizzo dell'Advertiser. Il campo TxAdd dell'header se settato a 1 indica che l'indirizzo del campo AdvA del payload è casuale, se settato a 0 esso è pubblico;

\begin{figure}[H]
\includegraphics[width=150pt]{adv_scan_ind_payload}
\centering
\caption{Payload di un PDU di tipo ADV\_ SCAN\_ IND }
\label{adv_scan_ind_payload}
\end{figure}

Il campo AdvA contiene l'indirizzo dell'Advertiser, sia esso pubblico o privato, ed il campo AdvData che contiene le eventuali informazioni da condividere ed ha una lunghezza massima di 31 Byte.

\end{itemize}

\subsubsection{Scanning PDU}

\begin{itemize}

\item SCAN\_ REQ: dopo aver ricevuto un pacchetto di Advertise, uno scanner in stato attivo può richiedere eventuali informazioni aggiuntive con questo tipo di PDU. Il campo TxAdd dell'header se settato a 1 indica che l'indirizzo del campo AdvA del payload, figura \ref{scan_req_payload} è casuale, se settato a 0 esso è pubblico; il campo RxAdd dell'header se settato a 1 indica che l'indirizzo del campo InitA del payload è casuale, se settato a 0 esso è pubblico.

\begin{figure}[H]
\includegraphics[width=150pt]{scan_req_payload}
\centering
\caption{Payload di un PDU di tipo SCAN\_ REQ }
\label{scan_req_payload}
\end{figure}

Nel campo ScanA viene indicato l'indirizzo del device che genera questo pacchetto di SCAN\_ REQ, mentre nel campo AdvA viene indicato l'indirizzo del dispositivo in stato di Advertise a cui è destinato questo pacchetto.

\item SCAN\_ RSP: dopo aver ricevuto un pacchetto di tipo SCAN\_ REQ destinato a lui, l'Advertiser può decidere di adempire alla richiesta inviando questo tipo di PDU con le informazioni aggiuntive che vuole fornire. Il campo TxAdd dell'header se settato a 1 indica che l'indirizzo del campo AdvA del payload, figura \ref{scan_rsp_payload} è casuale, se settato a 0 esso è pubblico;

\begin{figure}[H]
\includegraphics[width=150pt]{scan_rsp_payload}
\centering
\caption{Payload di un PDU di tipo SCAN\_ RSP }
\label{scan_rsp_payload}
\end{figure}

Nel campo AdvA viene indicato l'indirizzo del device che genera questo pacchetto di SCAN\_ RSP, mentre nel campo ScanRspData viene inserita l'eventuale informazione aggiuntiva da condividere, con lunghezza massima di 31 Byte.

\end{itemize}

\subsubsection{Initiating PDU}

\begin{itemize}
\item CONNECTION\_ REQ: inviata da un dispositivo in stato di scanning, consente a quest'ultimo di effettuare una connessione con il dispositivo in stato di Advertising che ha dichiarato la sua disponibilità ad una connessione.



\end{itemize}


\subsection{Bit Ordering}




