%!TEX root = Tesi.tex

\section{Bluetooth Low Energy}
\subsection{Introduzione}
Il Bluetooth Low Energy (BLE) è una tecnologia wireless introdotta nello standard 4.0, disegnata e commercializzata dal SIG\footnote{Bluetooth Special Interest Group: organizzazione che regolamenta e definisce gli standard  per la trasmissione dati tramite la tecnologia Bluetooth.}; ideata per trasmissioni di dati di breve dimensione, essa si differenzia dal Bluetooth classico per un inferiore consumo di energia, mantenendo la stessa portata trasmissiva.

\subsection{Banda Trasmissiva}
BLE opera alla stessa banda di frequenza del bluetooth classico, 2.400 - 2.4835 GHz, ma utilizza un numero inferiore di canali, 40 canali da 2 MHz l'uno.
Nel canale trasmissivo i dati sono trasmessi con una modulazione GFSK.\footnote{Gaussin Frequency Shift Keying, limita la larghezza dello spettro trasmissivo tramite un filtro di tipo Gaussiano.}
I bitrate trasmissivi sono limitati ai valori 125 kbit/s, 1 Mbit/s o 2 Mbit/s 
Bluetooth Low Energy utilizza il \emph{Frequency Hopping}, una tecnica trasmissiva che consiste nel cambiare canale trasmissivo per ogni pacchetto inviato durante una trasmissione; l'utilizzo di questa tecnica permette di ridurre i problemi di interferenza con gli altri canali ed allo stesso tempo di aumentare la sicurezza delle trasmissioni.

