%!TEX root = Tesi.tex

\section{Bluetooth Low Energy}
\subsection{Introduzione}
Il Bluetooth Low Energy (BLE) è una tecnologia wireless introdotta nello standard 4.0, disegnata e commercializzata dal SIG\footnote{Bluetooth Special Interest Group: organizzazione che regolamenta e definisce gli standard  per la trasmissione dati tramite la tecnologia Bluetooth.}; ideata per trasmissioni di dati di breve dimensione, essa si differenzia dal Bluetooth classico per un inferiore consumo di energia, mantenendo la stessa portata trasmissiva.

\subsection{Banda Trasmissiva}
BLE opera alla stessa banda di frequenza del bluetooth classico, 2,400 - 2,4835 GHz, ma utilizza un numero inferiore di canali, 40 canali da 2 MHz l'uno.
Nel canale trasmissivo i dati sono trasmessi con una modulazione GFSK.\footnote{Gaussin Frequency Shift Keying, limita la larghezza dello spettro trasmissivo tramite un filtro di tipo Gaussiano.}
I bitrate trasmissivi sono limitati ai valori 125 kbit/s, 1 Mbit/s o 2 Mbit/s.

\subsubsection{Mapping tra canali e frequenze}
Il protocollo BLE utilizza 40 canali fisici differenti per la trasmissione, con una numerazione che va da 0 a 39. Ogni canale ha una frequenza centrata in $2402 + (k * 2)$ MHz, dove k assume il valore del canale considerato.
Questi 40 canali si dividono in:
\begin{itemize}
\item 3 canali 0, 12 e 39 centrati nelle frequenze 2402 MHz, 2426 MHz e 2480 MHz ed utilizzati per l'Advertise. La distribuzione di questi canali non è casuale: essi difatti sono i canali utilizzati da un dispositivo per far conoscere la propria presenza ai vicini, è quindi fondamentale che almeno uno di essi sia disponibile per la trasmissione, quindi privo di disturbi generati da altri sistemi che operano nella stessa banda trasmissiva. Sono stati distribuiti lontani tra di loro cercando di utilizzare tutta la banda disponibile per massimizzare la possibilità di avere un canale libero per la trasmissione. Un altro motivo della scelta è stato quello di poter facilitare la co-esistenza con la tecnologia Wifi, che utilizza la stessa banda per trasmettere, centrata principalmente nei canali Wifi 1, 6 e 11.
\item i 37 canali restanti sono utilizzati per la trasmissione di dati che vengono utilizzati solo a connessione avvenuta per lo scambio di informazioni.
\end{itemize} 


Per una miglior leggibilità ed un più facile utilizzo, si è scelto di mappare i canali reali con dei corrispondenti indici fittizi che vedono nelle numerazioni da 0 a 36 i canali data e 37, 38, 39 per i canali di Advertise;
\begin{itemize}
\item[-] canale fisico 0, di Advertise a frequenza 2402 MHz, mappato come canale 37.
\item[-] canale fisico 12, di Advertise a frequenza 2426 MHz, mappato come canale 38.
\item[-] canale fisico 39, di Advertise a frequenza 2480 MHz, mappato come canale 39. 
\item[-] canale fisico 1, di scambio dati a frequenza 2404 MHz, mappato come canale 0.
\item[-] \dots
\item[-] canale fisico 38, di scambio dati a frequenza 2478 MHz, mappato come canale 36.
\end{itemize}

% figura canali

% figura canali con i 3 wifi



Bluetooth Low Energy utilizza il \emph{Frequency Hopping}, una tecnica trasmissiva che consiste nel cambiare canale trasmissivo per ogni pacchetto inviato durante una trasmissione; l'utilizzo di questa tecnica permette di ridurre i problemi di interferenza con gli altri canali ed allo stesso tempo di aumentare la sicurezza delle trasmissioni.

\subsection{Stack protocollare}
Storicamente lo stack Bluetooth è diviso in 2 componenti fondamentali: il \emph{Controller} e l'\emph{Host}.
La parte di Controller è quella che si occupa di eseguire le componenti di basso livello dello stack necessarie per gestire i pacchetti scambiati a livello fisico e le loro temporizzazioni; la parte di Host comprende le componenti di alto livello tra cui profili e API\footnote{Application Program Interface: insieme di procedure utili a svolgere uno specifico compito}. La parte di Host, diversamente da quella di controllo, astrae dall'hardware ed è caratterizzata da una gestione meno rigida delle temporizzazioni.

L'\emph{HCI}\footnote{Host Controller Interface}, l'interfaccia tra Host e Controller si occupa di mettere in comunicazione queste due componenti fondamentali, utilizzando canali comunicativi quali UART o USB.
Se i due componenti sono montati su uno stesso chip, come nel caso dei SoC\footnote{System on a Chip: sistema completamente contenuto su un solo Chip}, allora l'interfaccia HCI è opzionale e può essere omessa.

\includegraphics[scale=0.5]{stack_ble.png}

\begin{itemize}
\item Physical Layer (PHY): è il livello fisico, ovvero l'etere; lavora nella banda di frequenza 2,4 GHz ISM\footnote{Industrial, Scientific and Medical: frequenze riservate alle applicazioni di radiocomunicazioni non commerciali, ovvero per uso industriale, scientifico e medico.} Si compone di 40 canali da 2 MHz ognuno suddivisi in 3 per advertising e 37 per lo scambio dati.

\item Link Layer (LL): si occupa della gestione della sequenza e della temporizzazione dei pacchetti scambiati. Dialoga con i nodi vicini e scambia informazioni quali i parametri di connessione e controllo di flusso.\linebreak 
\'E una macchina a stati che si compone di 5 stati: Standby, Advertising, Scanning, Initiating, Connection. 

\includegraphics[scale=0.5]{LL_states}

\begin{itemize}

\item Advertising: il dispositivo rende noto agli ascoltatori della propria presenza, indicando la disponibilità ad una connessione ed inviando alcune informazioni utili.

\item Scanning: il dispositivo è in ascolto di tutte le informazioni trasmesse sui canali di advertise.

\item Initiating: un dispositivo in ascolto individua un Advertise di suo interesse ed indica la sua volontà di connettersi.

\item Connection: due o più dispositivi sono connessi.

\item Standby: il dispositivo è in uno stato di attesa caratterizzato da un basso consumo energetico.

\end{itemize}

Lo stato di Scanning può essere attivo (richiede informazioni aggiuntive) o passivo; Lo stato Connection anch'esso si divide in 2 sottostati: Central e Peripheral.

\item Logical Link Control and Adaptation Protocol (L2CAP): fornisce servizi sui dati ai livelli superiori, come ad esempio il Security Manager Protocol o l'Attribute Protocol. \'E responsabile inoltre della segmentazione e ricostruzione dei pacchetti da e verso i livelli inferiori.

\item Security Manager (SM): è responsabile del pairing dei dispositivi e della distribuzione delle chiavi crittografiche; BLE utilizza lo standard AES-128 bit per la cifratura dei dati ed il sistema di pairing per la distribuzione delle chiavi.

\item Attribute Protocol (ATT): gestisce la distribuzione delle informazioni riguardanti le coppie attributo-valore presenti in un device peripheral (Server)

\item Generic Access Profile (GAP): controlla \emph{advertising} e \emph{connection} . Divide i dispositivi bluetooth low energy in:
\begin{itemize}
\item Peripheral: normalmente dispositivi dotati di una limitata capacità di calcolo, come ad esempio un sensore di temperatura.
\item Central: dispositivo che gestisce la rete bluetooth e che richiede una maggiore capacità di calcolo; mentre un dispositivo periferico può avere una connessione con un solo dispositivo centrale, un centrale può gestire più dispositivi periferici andando a creare una Piconet\footnote{Rete Bluetooth composta da massimo otto dispositivi in relazione master-slave e fino a 255 dispositivi in modalità inattiva o parcheggio.}
\end{itemize} 

\item Generic Attribute Profile (GATT): entra in gioco a connessione avvenuta, definisce il modo in cui 2 dispositivi BLE scambiano dati utilizzando i concetti di \emph{Services} e \emph{Characteristics}.

\begin{itemize}
\item GATT server: è un peripheral device, che tramite il protocollo ATT permette al central di conoscere i dati che ha memorizzato in strutture di tipo service-characteristic.
\item GATT client: è il central device, che gestisce le comunicazioni e che richiede ai server periferici i dati raccolti.
\end{itemize}

\end{itemize}



